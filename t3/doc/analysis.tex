\section{Theoretical Analysis}
\label{sec:analysis}

In this section, the eletrical device shown in Figure~\ref{fig:rc} is analysed
theoretically to predict the output of the Envelope Detector and Voltage Regulator circuits.

\section{Envelope Detector}

The circuit consists of a diode(that receives a sinusoidal signal) and a resistance in parallel with a capacitor. The
voltage source $v_S(t)$ drives its input, and the output voltage $v_3(t)$ is taken from
the capacitor terminals. Using the diode ideal model and the solution for the RC circuit, at $t=0$ the diode is on (its a short circuit) and $v_3(t)=v_S(t)$. When $t=tOFF$ the diode goes off (its an open circuit) and $v_3(t)=-R*i_{C}$.
Noticing that at $tOFF$ $i_{R}=-i_{C}$ then $tOFF = \frac{1}{w} \times \arctan{1*w*R*C}$ and for $t>tOFF$ $v_3(t)= A\times \sin(w\times tOFF)\times \exp{-\frac{t-tOFF}{RC}}$.
The time $tON$ is given by the first intersection between the solution for $t>tON$ and $v_S(t)$.

\begin{figure}[h] \centering
\includegraphics[width=0.5\linewidth]{venvlope.eps}
\caption{Output voltage of the envelope detector(v3) and the voltage regulator(v4 - DC Output)}
\label{fig:vevelope}
\end{figure}
\newpage
\section{Voltage Regulator}
Using the real diode model and applying the Kirchhoff Voltage Law (KVL), a single
equation for the single loop in the circuit can be written as

\begin{equation}
  v_{4}+R\times I_{s} \times \exp{\frac{v_{4}}{19\eta \times v_{T}}-1} - v_{3}=0
  \label{eq:kvl}
\end{equation}

Solving the equation with the Newton Raphson’s iterative method, we got the DC output (v4). And in the image below, we see the oscillation around the average. Noticing that it is a little bit different from the simulation result, as was expected.

\begin{figure}[h] \centering
\includegraphics[width=0.5\linewidth]{vripple.eps}
\caption{Ripple for the DC output}
\label{fig:vripple}
\end{figure}
