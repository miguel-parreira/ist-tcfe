\section{Theoretical Analysis}
\label{sec:analysis}

In this section, the eletrical device shown in Figure~\ref{fig:rc} is analysed
theoretically to predict the output of the Gain Stage and Output Stage circuits.

\section{Operating Point}

\subsection{Gain Stage}
The DC Analysis corresponds to a frequency of zero Hz which means that the capacitors are equivalent to an open circuit.
For the bias circuit, it was considered the Thévenin’s equivalent with $R_{eq1}=R_{1}||R_{2}$ and $V_{eq1}=\frac{R_{2}}{R_{1}+R_{2}}\times V_{CC}$
Using KVL we get:
\begin{equation}
  V_{eq1} + R_{eq}\times I_{B1} + V_{BEON} + R_{E}\times I_{E1}  = 0
  \label{eq:kvl}
\end{equation}
From the transistor model:
\begin{equation}
  I_{E1} = (1+\beta_{F})\times I_{B1}
  \label{eq:IE}
\end{equation}

\begin{equation}
  I_{C1} = \beta_{F}\times I_{B1}
  \label{eq:IE}
\end{equation}

Using the ohm's law and the KVL in the other loop we get the voltages:

\input{tabela1.tex}

Comparing the theoretical values with the simulation, they are satisfyingly close.
\subsection{Output Stage}
Using the same methods:

\begin{equation}
  V_{I2} - V_{CC} + V_{EBON} + R_{E}\times I_{E}  = 0
  \label{eq:kvl}
\end{equation}
\begin{equation}
  V_{O} = V_{CC} - R_{E}\times I_{E}
  \label{eq:VO}
\end{equation}

We get the following results:

\input{tabela2.tex}

Comparing the theoretical values with the simulation, they are pleasantly close.

\section{Gain}
In the these and the following setions we are going to use the incremental model for the transistors and having in mind that the capacitors operate as approximately a short circuit for higher frequencies.
\subsection{Gain Stage}
Using another Thévenin’s equivalent for the source signal and the bias circuit, we get $R_{eq2}=R_{eq1}||R_{S}$ and  $v_{eq2}=\frac{R_{eq1}}{R_{eq1}+R_{S}}\times v_{I}$.
Keeping in mind that the incremental parameters are given by $R_{\pi1}=\frac{\beta_{F}}{g_{m1}}$, $g_{m1}=\frac{I_{C}}{V_{T}}$  and $R_{o1}\approx \frac{V_{A}}{I_{C}}$

And doing mesh analysis and making $R_{E}=0$(because of the bypass capacitor), the gain is given by:
\begin{equation}
  AV_{1}= \frac{R_{eq1}}{R_{eq1}+R_{S}}\times R_{C} \times \frac {R_{E} - g_{m1} R_{\pi1} R_{o1} }{(R_{o1}+R_{C}+R_{E})\times(R_{eq2}+R_{\pi1}+R_{E})+g_{m1} R_{E} R_{\pi1} R_{o1} - R_{E}^2}=41.626dB
  \label{eq:Gain1}
\end{equation}


\subsection{Output Stage}
Aplying KCl, we get directly:

\begin{equation}
  AV_{2}= \frac{g_{m2}}{g_{\pi2}+g_{out}+g_{o2}+g_{m2}}=0.98724
  \label{eq:Gain1}
\end{equation}

Where $g_{\pi2}=\frac{1}{R_\pi2}$, $g_{out}=\frac{1}{R_{out}}$ and $g_{o2}=\frac{1}{R_{o2}}$

\subsection{Total Circuit}
In this calculation the whole circuit is taken into consideration, simplifying the Gain Stage.
\begin{equation}
  AV= \frac{g_{B}+\frac{g_{m2}*g_{B}}{g_{\pi2}}}{g_{B}+g_{out}+g_{o2}+\frac{g_{m2}*g_{B}}{g_{\pi2}}} \times AV_{1}=41.173
  \label{eq:Gain1}
\end{equation}



\section{Input impedance}
This is calculated by imposing a voltage in the imput terminals and measuring the current through it we get the imput impedance by dividing them $Z_{I}=\frac{v_{i}}{I}$.
\subsection{Gain Stage}
Doing mesh analysis and solving for the wanted current, we get($R_E=0$):

\begin{equation}
  Z_{I1} = \frac{1}{(\frac{1}{R_{B}}+\frac{1}{((((r_{o1}+R_{C1}+R_{E})*(r_{\pi1}+R_{E})+g_{m1}*R_{E}*r_{o1}*r_{\pi1} - R_{E}^2)/(r_{o1}+R_{C1}+R_{E})))}}=445.51\si{\ohm}
  \label{eq:imputimpe1}
\end{equation}

This value is sort of compatible because $Z_{I1}$ is 4.5x greater than $R_{S}$.

\subsection{Output Stage}
Using the gain expression and doin a simple ohm's law, we get

\begin{equation}
  Z_{I2} = \frac{(g_{m2}+g_{\pi2}+g_{o2}+g_{out})}{(g_{\pi2})(g_{\pi2}+g_{o2}+g_{e2})}=5595.5\si{\ohm}
  \label{eq:imputimpe2}
\end{equation}

\subsection{Total Circuit}
We considered it equal to $Z_{I}=Z_{I1}=445.51\si{\ohm}$.

\section{Output impedance}
This is calculated by shutting off the imput source and imposing a voltage in the output terminals and measuring the current through it, we get the output impedance by dividing them $Z_{O}=\frac{v_{O}}{I}$

\subsection{Gain Stage}

Looking at the circuit we notice that $v_{\pi1}=0$ which means that the impedance seen by the test source is simply:

\begin{equation}
    Z_{O1} = \frac{1}{\frac{1}{R_{o1}}+\frac{1}{R_{C}}}=378.82\si{\ohm}
  \label{eq:imputimpe2}
\end{equation}

This value is compatible with $Z_{I2}$ because $Z_{I2}$ is much greater.

\subsection{Output Stage}
Applying KCL and noticing that $v_{\pi2}=-v_{o}$:
\begin{equation}
    Z_{O2} = \frac{1}{(g_{m2}+g_{\pi2}+g_{o2}+g_{out})}=0.31008\si{\ohm}
  \label{eq:imputimpe2}
\end{equation}

\subsection{Total Circuit}

\begin{equation}
    Z_{O} = \frac{1}{g_{o2}+\frac{g_{m2}*g_{B}}{g_{\pi2}}+g_{out}+g_{B}}=1.8718\si{\ohm}
  \label{eq:imputimpe2}
\end{equation}

The total output impedance is compatible enough to connect to an $8\si{\ohm}$ load.

\section{Frequency response}

Using the complete incremental model  for the circuit (with the capacitors on), applying the node method and solving ($AV=\frac{V_{o}}{V_{i}}$) for a $10Hz$ frquency up to $100MHz$, we get the following results:

\begin{figure}[h] \centering
\includegraphics[width=0.6\linewidth]{gain.eps}
\caption{Frequency Response}
\label{fig:Freqresp}
\end{figure}
