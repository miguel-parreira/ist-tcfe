\section{Simulation Analysis}
\label{sec:simulation}

\subsection{Operating Point Analysis}


Table~\ref{tab:op} shows the simulated operating point results for the circuit
under analysis.

\begin{table}[h]
  \centering
  \begin{tabular}{|l|r|}
    \hline
    {\bf Name} & {\bf Value [A or V]} \\ \hline
    \input{../sim/op_tab}
  \end{tabular}
  \caption{Operating point. A variable preceded by @ is of type {\em current}
    and expressed in Ampere; other variables are of type {\it voltage} and expressed in
    Volt.}
  \label{tab:op}
\end{table}



\subsection{Transient Analysis}

Figures~\ref{fig:transvin},~\ref{fig:transvo1} and ~\ref{fig:transvo2} show the simulated transient analysis results for the circuit under analysis.

\begin{figure}[h] \centering
\includegraphics[width=0.6\linewidth]{vin.eps}
\caption{Transient input voltage}
\label{fig:transvin}
\end{figure}

\begin{figure}[h] \centering
\includegraphics[width=0.6\linewidth]{vo1.eps}
\caption{Transient output voltage, Gain Stage}
\label{fig:transvo1}
\end{figure}

\begin{figure}[h] \centering
\includegraphics[width=0.6\linewidth]{vo2.eps}
\caption{Transient output voltage}
\label{fig:transvo2}
\end{figure}


\subsection{Frequency Analysis}

\subsubsection{Magnitude Response}

Figures~\ref{fig:MRGS} and ~\ref{fig:MROS} show the magnitude of the frequency response for the
circuit under analysis.

\begin{figure}[h] \centering
\includegraphics[width=0.6\linewidth]{vo1f.eps}
\caption{Magnitude response, Gain Stage}
\label{fig:MRGS}
\end{figure}

\begin{figure}[h] \centering
\includegraphics[width=0.6\linewidth]{vo2f.eps}
\caption{Magnitude response, Output Stage}
\label{fig:MROS}
\end{figure}


\subsubsection{Input Impedance}
The simulation gives an input impedance of $555.9097 \ohm$ which is very good compared to $R_{in}=100\ohm$ .

\subsubsection{Output Impedance}
The simulation gives an output impedance of $4.2725 \ohm$ which is reasonable compared to $R_{L}=8\ohm$ .
