\section{Theoretical Analysis}
\label{sec:analysis}

In this section, the circuit shown in Figure~\ref{fig:rc} is analysed
theoretically, in terms of node, AC and phasor analysis.

\subsection{Circuit solution for $t<0$}

After naming the nodes from 1 to 8, as shown in Figure~\ref{fig:rc}, assigning the $4^{th}$ node as the reference node ($V_{4}$ =GND) and assuming that the circuit is open in the capacitor branch(because it's supposed that the capacitor is totally charged in $t->0$) the equation to some of the nodes(those that aren't connected to Voltage sources) can be derived from the  Kirchhoff Current Law (KCL).
For the nodes 2,3,6 and 7, respectively:

\begin{equation}
  (V_2-V_5)G_3 + (V_2-V_1)G_1 + (V_2-V_3)G_2 = 0
  \label{eq:Node2}
\end{equation}

\begin{equation}
  (V_3-V_2)G_2 - K_b(V_2-V_5)= 0
  \label{eq:Node3}
\end{equation}

\begin{equation}
  (V_5-V_6)G_5 - K_b(V_2-V_5) = 0
  \label{eq:Node6}
\end{equation}

\begin{equation}
  (V_4-V_7)G_6 - (V_7-V_8)G_7 = 0
  \label{eq:Node7}
\end{equation}

Now looking at the nodes that are connected to Voltage Sources and knowing the definition of voltage difference, the equations are:

\begin{equation}
  V_1 = V_S
  \label{eq:Node1}
\end{equation}

\begin{equation}
  V_5= K_d \times G_6(V_4-V_7) + V_8
  \label{eq:Node8}
\end{equation}

Noticing that the current through $V_d$ is the same as the one through $R_7$, the equation to supernode 5-8 goes as:

\begin{equation}
  (V_5-V_4)G_4 + (V_5-V_6)G_5 + (V_5-V_2)G_3 + (V_8-V_7)G_7 = 0
  \label{eq:Node5}
\end{equation}

Solving the linear system:

%%MATRIZES%
$$
\begin{bmatrix}
1  &  0 & 0 & 0 & 0 & 0 & 0 & 0       \\
-G_1 & G_1+G_2+G_3 & -G_2 & 0 & -G_3 & 0 & 0 & 0      \\
0 & -K_b-G_2 & G_2 & 0 & K_b & 0 & 0 & 0\\
0 & 0 & 0  & 1 & 0 & 0 & 0 & 0      \\
0 & -G_3 & 0 & -G_4 & G_4+G_5+G_3 & -G_5 & -G_7 & G_7\\
0 & -K_b & 0 & 0 & G_5+K_b & -G_5 & 0 & 0 \\
0 & 0& 0 & G_6 & 0 & 0 & -G_6-G_7 & G_7\\
0 & 0 & 0 & -K_d\times G_6 & 1 & 0 & K_d\times G_6 & -1\\
\end{bmatrix}
\begin{bmatrix}
V_1     \\
V_2    \\
V_3   \\
V_4     \\
V_5     \\
V_6     \\
V_7     \\
V_8     \\
\end{bmatrix}
=
\begin{bmatrix}
V_S   \\
0    \\
0  \\
0  \\
0  \\
0  \\
0  \\
0  \\
\end{bmatrix}
\quad
$$

The results are:

%%%%%%%%TABELA%%%%%%%%%%%%%%%%%%%%%%

\input{tabela1.tex}
\input{tabela2.tex}

 %%%%%%%%%%%%%%%%%%%%%%%%%%%%%%%%%%%%%%%%%%%%%%%%%%%%%%%%%%%

\subsection{Circuit solution for $t=0$}

To calculate  the equivalent resistance, we made $V_s=0$ (in $t=0$, $V_s\neq0$ but it changes in $t_0+dt$ so it's neglected) and replaced the capacitor with a voltage source $V_x= V_6-V_8$, where $V_6$ and $V_8$ are the voltages in nodes 6 and 8 as obtained in $t<0$. With the nodal analysis, the current Ix supplied by Vx was calculated. Finally, the equivalent resistor as Req = Vx/Ix,and the time constant were computed.
This procedure was executed so that we could get the time constant for the RC circuit, solve the natural solutions for any variable and get the voltages in $t=0$.
We used the same equations as in $t<0$ with the following changes:

\begin{equation}
  (V_5-V_4)G_4 + (V_5-V_6)G_5 + (V_5-V_2)G_3 + (V_8-V_7)G_7 = I_x
  \label{eq:Node5.2}
\end{equation}

\begin{equation}
  (V_5-V_6)G_5 - K_b(V_2-V_5) = I_x
  \label{eq:Node6.2}
\end{equation}

\begin{equation}
  V_8 = V_6 + V_x
  \label{eq:Equation9}
\end{equation}




Solving the linear system:
%%MATRIZES%
$$
\begin{bmatrix}
1  &  0 & 0 & 0 & 0 & 0 & 0 & 0 &0      \\
-G_1 & G_1+G_2+G_3 & -G_2 & 0 & -G_3 & 0 & 0 & 0   &0   \\
0 & -K_b-G_2 & G_2 & 0 & K_b & 0 & 0 & 0&0\\
0 & 0 & 0  & 1 & 0 & 0 & 0 & 0  &0    \\
0 & -G_3 & 0 & -G_4 & G_4+G_5+G_3 & -G_5 & -G_7 & G_7 & -1\\
0 & -K_b & 0 & 0 & G_5+K_b & -G_5 & 0 & 0 & -1\\
0 & 0& 0 & G_6 & 0 & 0 & -G_6-G_7 & G_7& 0\\
0 & 0 & 0 & -K_d\times G_6 & 1 & 0 & K_d\times G_6 & -1& 0\\
0 & 0 & 0 & 0 & 0 & -1 & 0 & 1 & 0\\
\end{bmatrix}
\begin{bmatrix}
V_1     \\
V_2    \\
V_3   \\
V_4     \\
V_5     \\
V_6     \\
V_7     \\
V_8     \\
I_x     \\
\end{bmatrix}
=
\begin{bmatrix}
0   \\
0    \\
0  \\
0  \\
0  \\
0  \\
0  \\
0  \\
V_x \\
\end{bmatrix}
\quad
$$

The results are:
\input{tabela3.tex}

\subsection{Natural Solution for Node 6 - $t>0$}

The natural solution for the voltage in node 6 is given by the general expression for RC and RL Circuits:

\begin{equation}
    x(t)=x(\infty) + [x(0)-x(\infty)]\exp(\frac{-t}{\tau})
\end{equation}

Where $x(t)=v_{6n}(t)$, $x(\infty)=0$ because the capacitor discharges and $x(0)=v_{6n}(0)$ calculated in the subsection before.

\begin{figure}[h] \centering
\includegraphics[width=0.5\linewidth]{v6_natural.eps}
\caption{Natural Solution for node 6}
\label{fig:forced}
\end{figure}

\subsection{Forced Solution - Phasors ($t>0$)}

Choosing the input frequency of $f=1Khz$ and making use of the phasors corresponding to each variable in the circuit: $\tilde{V}_{s}=1$, $\tilde{Z}_{c}=\frac{1}{j\times C \times 2\pi \times f}$

Running node analysis:
%%MATRIZES%
$$
\begin{bmatrix}
1  &  0 & 0 & 0 & 0 & 0 & 0 & 0       \\
-G_1 & G_1+G_2+G_3 & -G_2 & 0 & -G_3 & 0 & 0 & 0      \\
0 & -K_b-G_2 & G_2 & 0 & K_b & 0 & 0 & 0\\
0 & 0 & 0  & 1 & 0 & 0 & 0 & 0      \\
0 & -G_3 & 0 & -G_4 & G_4+G_5+G_3 & -G_5+(\frac{1}{Z_{c}}) & -G_7 & G_7-(\frac{1}{Z_{c}})\\
0 & -K_b & 0 & 0 & G_5+K_b & -G_5+(\frac{1}{Z_{c}}) & 0 & -(\frac{1}{Z_{c}}) \\
0 & 0& 0 & G_6 & 0 & 0 & -G_6-G_7 & G_7\\
0 & 0 & 0 & -K_d\times G_6 & 1 & 0 & K_d\times G_6 & -1\\
\end{bmatrix}
\begin{bmatrix}
\tilde{V}_{1}     \\
\tilde{V}_{2}    \\
\tilde{V}_{3}   \\
\tilde{V}_{4}     \\
\tilde{V}_{5}     \\
\tilde{V}_{6}     \\
\tilde{V}_{7}     \\
\tilde{V}_{8}     \\
\end{bmatrix}
=
\begin{bmatrix}
1   \\
0    \\
0  \\
0  \\
0  \\
0  \\
0  \\
0  \\
\end{bmatrix}
\quad
$$

The results are:
\input{tabela4.tex}


Specifically for node 6:

\begin{figure}[h] \centering
\includegraphics[width=0.7\linewidth]{v6_forçada.eps}
\caption{Forced Solution for node 6}
\label{fig:forced}
\end{figure}



\subsection{Total Solution - $t>0$}

Superimposing the natural and forced solutions we get the total solution for $v_{6}(t)$.
Plotting $v_{6}(t)$ and $v_{s}(t)$:

\begin{figure}[h] \centering
\includegraphics[width=0.7\linewidth]{v6.eps}
\caption{Total Solution for node 6}
\label{fig:total}
\end{figure}


\subsection{Frequency Analysis}

Using the last system of equations, we solved the phasors in function of the frequency for $0.1Hz<f<1MHz$, getting $v_{6}(f)$, $v_{s}(f)$ and $v_{c}(f)=v_{6}(f)-v_{8}(f)$

\begin{figure}[h] \centering
\includegraphics[width=0.6\linewidth]{bode.eps}
\caption{Frequency Analysis}
\label{fig:frequency}
\end{figure}

Comparing the magnitude, we see FALTA ESCREVER
