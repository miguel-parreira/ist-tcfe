\section{Simulation Analysis}
\label{sec:simulation}

\subsection{Operating Point Analysis for $t<0$}

Table~\ref{tab:op} shows the simulated operating point results for the circuit
under analysis. Compared to the theoretical analysis results, one notices the
following differences: none.

\begin{table}[h]
  \centering
  \begin{tabular}{|l|r|}
    \hline
    {\bf Name} & {\bf Value [A or V]} \\ \hline
    \input{../sim/op_tab}
  \end{tabular}
  \caption{Operating point. A variable preceded by @ is of type {\em current}
    and expressed in Ampere; other variables are of type {\it voltage} and expressed in
    Volt.}
  \label{tab:op}
\end{table}

\subsection{Operating Point Analysis for $t=0$}

For the purpose of comparing with the theoretical analysis, the simulated operating point results in:

\begin{table}[h]
  \centering
  \begin{tabular}{|l|r|}
    \hline
    {\bf Name} & {\bf Value [A or V]} \\ \hline
    \input{../sim/op2_tab}
  \end{tabular}
  \caption{Operating point. A variable preceded by @ is of type {\em current}
    and expressed in Ampere; other variables are of type {\it voltage} and expressed in
    Volt.}
  \label{tab:op2}
\end{table}




\subsection{Transient Analysis - $t>0$}
\subsubsection{Natural Response}

Figure~\ref{fig:trans} shows the simulated transient analysis results for the
circuit under analysis in the natural state ($V_{s}=0$). The boundary conditions were set by the command .ic for the nodes 6 and 8. Compared to the theoretical analysis results, one notices no differences:

\begin{figure}[h] \centering
\includegraphics[width=0.4\linewidth]{trans.eps}
\caption{Transient output voltage}
\label{fig:trans}
\end{figure}

\subsubsection{Total Response}

Figure~\ref{fig:trans4} shows the simulated transient analysis results for the
circuit under analysis in the stimulated state ($V_{s}=1$ - AC). The boundary conditions were set by the command .ic for the nodes 6 and 8. Compared to the theoretical analysis results, one notices no differences:

\begin{figure}[h] \centering
\includegraphics[width=0.3\linewidth]{trans4.eps}
\caption{Transient output voltage}
\label{fig:trans4}
\end{figure}


\subsection{Frequency Analysis}

\subsubsection{Magnitude Response}

Figure~\ref{fig:acm} shows the magnitude of the frequency response for the
circuit under analysis. Compared to the theoretical analysis results, one
notices no differences:

\begin{figure}[h] \centering
\includegraphics[width=0.3\linewidth]{acm.eps}
\caption{Magnitude response}
\label{fig:acm}
\end{figure}



\subsubsection{Phase Response}

Figure~\ref{fig:acp} shows the magnitude of the frequency response for the
circuit under analysis. Compared to the theoretical analysis results, one
notices the following differences: they are symmetric.

\begin{figure}[h] \centering
\includegraphics[width=0.3\linewidth]{acp.eps}
\caption{Phase response}
\label{fig:acp}
\end{figure}
