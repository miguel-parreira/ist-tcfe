\section{Theoretical Analysis}
\label{sec:analysis}

In this section, the circuit shown in Figure~\ref{fig:rc} is analysed
theoretically, in terms of node and mesh analysis.

\subsection{Mesh Analysis Method}

The circuit consists of four meshes named A,B,C and D. Looking at mesh A and considering the Kirchhoff Voltage Law (KVL), an equation for the single loop in the circuit can be written as:

\begin{equation}
  V_a= R_1(I_A) + R_3(I_A + I_B) + R_4(I_A + I_C)
  \label{eq:MeshA}
\end{equation}

In the calculation of voltage drop in $R_{3}$, the total current through $R_{3}$ is a combination of the currents in mesh A and mesh B. The same goes to $R_{4}$ and any element that is common to two meshes.

Continuing to mesh C, the equation is:
\begin{equation}
  V_c= K_c\times I_C= R_4(I_A + I_C) + R_6(I_C) + R_7(I_C)
  \label{eq:MeshC}
\end{equation}

Noticing that $I_{C}=I_{c}$.

In the other two meshes, B and D, the current sources impose the current of each mesh. That being said, the equation for mesh B is:

\begin{equation}
  I_B= K_b\times V_b = K_b\times R_3(I_A + I_B)
  \label{eq:MeshB}
\end{equation}

And for mesh D is:

\begin{equation}
  I_D= I_d
  \label{eq:MeshD}
\end{equation}

Solving the linear system:

$$
\begin{bmatrix}
R_1+R_3+R_4       &   R_3       & R_4     &   0       \\
K_b \times R_3       &   K_b \times R_3-1       & 0    &   0      \\
R_4  &  0  &   R_4+R_6+R_7-K_c  &   0 \\
0       &   0      & 0    &   1      \\
\end{bmatrix}
\begin{bmatrix}
I_A     \\
I_B    \\
I_C   \\
I_D     \\
\end{bmatrix}
=
\begin{bmatrix}
V_a   \\
0    \\
0  \\
I_d   \\
\end{bmatrix}
\quad
$$


The results are:

%%%%%%%%TABELA%%%%%%%%%%%%%%%%%%%%%%
\begin{table}[h]


\label{tab:tables}
\begin{center}
\begin{tabular}{|c|c|}
 \hline
  Mesh Current & Value\hspace{1mm}(mA)\\
 \hline
 $I_{A}$ & 1.809244100051636 \\
 \hline
 $I_{B}$ &-1.809325173903972\\
 \hline
 $I_{C}$ &-1.040984044022358\\
 \hline
 $I_{D}$ &1.017457962400000\\
 \hline
\end{tabular}
\caption{Solutions to Mesh Analysis Method}
\label{table:tab2}
\end{center}
\end{table}
 %%%%%%%%%%%%%%%%%%%%%%%%%%%%%%%%%%%%%%%%%%%%%%%%%%%%%%%%%%%

\subsection{Node Analysis Method}

After naming the nodes from 1 to 8, as shown in Figure~\ref{fig:rc}, and assigning the $8^{th}$ node as the reference node ($V_{8}$ =G ND), the equation to some of the nodes(those that aren't connected to Voltage sources) can be derived from the  Kirchhoff Current Law (KCL).
For the nodes 2,3,6 and 7, respectively:

\begin{equation}
  (V_1-V_2)G_1 + (V_3-V_2)G_2 - (V_2-V_5)G_3 = 0
  \label{eq:Node2}
\end{equation}

\begin{equation}
  K_b(V_2-V_5) - (V_3-V_2)G_2 = 0
  \label{eq:Node3}
\end{equation}

\begin{equation}
  (V_5-V_6)G_5 + I_d - K_b(V_2-V_5) = 0
  \label{eq:Node6}
\end{equation}

\begin{equation}
  (V_4-V_7)G_6 - (V_7-V_8)G_7 = 0
  \label{eq:Node7}
\end{equation}

Now looking at the nodes that are connected to Voltage Sources and knowing the definition of voltage difference, the equations are:

\begin{equation}
  V_1 = V_a + V_4
  \label{eq:Node1}
\end{equation}

\begin{equation}
  V_5= K_C \times G_6(V_4-V_7) + V_8
  \label{eq:Node5}
\end{equation}

Noticing that the current through $V_a$ is the same as the one through $R_1$, the equation to node 4 goes as:

\begin{equation}
  -(V_4-V_7)G_6 + (V_5-V_4)G_4 - (V_1-V_2)G_1 = 0
  \label{eq:Node5}
\end{equation}

Solving the linear system:
%%MATRIZES%
$$
\begin{bmatrix}
1  &  0 & 0 & -1 & 0 & 0 & 0 & 0       \\
G_1 & -G_1-G_2-G_3 & G_2 & 0 & G_3 & 0 & 0 & 0      \\
0 & K_b+G_2 & -G_2 & 0 & -K_b & 0 & 0 & 0\\
-G_1 & G_1 & 0  & -G_6-G_4 & G_4 & 0 & G_6 & 0      \\
0 & 0 & 0 & K_c \times G_6 & -1 & 0 & -K_c \times G_6 & 1 \\
0 & -K_b & 0 & 0 & G_5+K_b & -G_5 & 0 & 0 \\
0&0&0&G_6&0&0&-G_6-G_7&G_7\\
0 & 0 & 0 & 0 & 0 & 0 & 0 & 1\\
\end{bmatrix}
\begin{bmatrix}
V_1     \\
V_2    \\
V_3   \\
V_4     \\
V_5     \\
V_6     \\
V_7     \\
V_8     \\
\end{bmatrix}
=
\begin{bmatrix}
V_a   \\
0    \\
0  \\
0  \\
0  \\
-I_d   \\
0  \\
0  \\
\end{bmatrix}
\quad
$$
The results are:

%%%%%%%%TABELA%%%%%%%%%%%%%%%%%%%%%%
\begin{table}[h]


\label{tab:tables}
\begin{center}
\begin{tabular}{|c|c|}
 \hline

  Node Voltage & Value\hspace{1mm}(V)\\
 \hline
 $V_{1}$ & 1.807255180205026 \\
  \hline
 $V_{2}$ & -8.810805588482041e-03\\
  \hline
 $V_{3}$ & -3.753133034617004\\
  \hline
 $V_{4}$ & -3.232675342464974\\
  \hline
 $V_{5}$ & -8.558265289355584e-03\\
  \hline
 $V_{6}$ & 8.744404418221231\\
  \hline
 $V_{7}$ & -1.047445772479076\\
  \hline
 $V_{8}$ & 0.000000000000000 \\


 \hline
\end{tabular}
\caption{Solutions to Mesh Analysis Method}
\label{table:tab2}
\end{center}
\end{table}
 %%%%%%%%%%%%%%%%%%%%%%%%%%%%%%%%%%%%%%%%%%%%%%%%%%%%%%%%%%%
